% -*- mode: Latex -*-
% Time-stamp: "2015-02-18 17:55:40 sb"

\documentclass[twocolumn,aps,pra,showpacs,preprintnumbers,bibnotes]{revtex4-1}

\usepackage[T1]{fontenc}

\usepackage{natbib}
\usepackage{graphicx}
\usepackage{bm}
\usepackage{color}
\usepackage{amsmath}

%% Keep track of changes
%\usepackage[final]{changes}
\usepackage[draft]{changes}
\definechangesauthor[color=green]{AM}
\definechangesauthor[color=blue]{SB}

\DeclareMathOperator{\re}{Re}
\DeclareMathOperator{\im}{Im}

\newcommand\avg[1]{\left\langle#1\right\rangle}

\newcommand\unit[2]{\ensuremath{#1~\mathrm{{#2}}}}

\newcommand\Ket[1]{\ensuremath{|{#1}\rangle}}
\newcommand\Bra[1]{\ensuremath{\langle{#1}|}}

\newcommand\Isotope[2]{\ensuremath{^{#1}\mathrm{#2}}}
\newcommand\Li{\Isotope{6}{Li}}
\newcommand\K{\Isotope{40}{K}}
\newcommand\Rb{\Isotope{87}{Rb}}
\newcommand\Sr{\Isotope{87}{Sr}}
\newcommand\Yb{\Isotope{171}{Yb}}
\newcommand\Er{\Isotope{167}{Er}}
\newcommand\Dy{\Isotope{161}{Dy}}

\renewcommand{\vec}[1]{\ensuremath{\bm{#1}}}

\newcommand\NA{\ensuremath{\mathrm{NA}}}
\newcommand{\osim}{\ensuremath{\mathord{\sim}}}
\newcommand{\Erecl}{\ensuremath{E_\mathrm{rec}}}
\newcommand{\Ereca}{\ensuremath{E_\mathrm{rec}^\mathrm{(a)}}}
\newcommand\FIXME{{\color{red}\ensuremath{\mathrm{FIXME}}}}

\newcommand\TrapFreq{\ensuremath{\nu_t}}
\newcommand\RecoilEnergy{\ensuremath{E_\mathrm{rec}}}


\begin{document}

\title{Design and Implementation of Low }


\author{Li Microscope Collective}
\author{M. Greiner}
\affiliation{
  Department of Physics, Harvard University,
  Cambridge, Massachusetts, 02138, USA}

\date{\today}
\begin{abstract}
We describe the design of a stable high power 1064 nm laser for use in optical lattice experiments. The system is based on a high quality mephisto NPRO seeding an array of four heavily modified fiber amplifiers. The intensity of every beam is stabilized with a low noise, nonlinear feedback system, and the waist can be smoothly controlled. 
\end{abstract}
\maketitle
\section{Introduction}
Trapping ultracold atoms in an array of optical traps, known as an optical lattice, is a powerful technique for manipulation and control of quantum many-body systems. 
It has facilitated precision measurements, engineering of interesting many body states, and is a possible quantum computing platform.
However, a dominant trend in cold atom experiments has been the need to attain colder temperatures at energy scales approaching the Hz level.
Such experiments are limited by heating rates arising from positional, phase and intensity noise of the optical lattice. 

By FIGURE - high level block diagram of system

\begin{itemize}
	\item requirements 
	\item what have we built
\end{itemize}

We demonstrate high power lattice 


\section{Mephisto}
The lattice laser system is seeded by an Innolight Mephisto laser, which provides approximately 2 W of 1064 nm light. 
This commercial laser utilizes a non-planar ring oscillator (NPRO) design that ensures superb phase and intensity stability. This source is split into four fibers (Thorlabs XXX) by means of polarizing beam splitters, where two of the fibers are detuned from each other by XXX MHz, to prevent interference of the lattice beams. Figure 1 shows the mechanical layout of this system, as well as the intensity noise performance (see supplementary material).

\section{Nufern Fiber Amplifiers}
Quantum gas microscopy of Li requires single site trap frequencies of approximately 1 MHz \cite{single site paper}.
Such trap frequencies may only be attained using tightly focused high power beams, which requires that the seed power be amplified to approximately 30 W per lattice axis, which is done using Nufern Fiber amplifiers (PART NUMBER XXX).
These amplifiers can amplify 150 mW to up to approximately 45 W (depending on pump diode current) by using a length of fiber doped with XXX (erbium or ytterbium) and pumped with xxx nm light from fiber coupled diode sources to provide a gain medium for 1064 nm light. 
The amplification is performed using a two stage amplifier design, where the stages are contolled using separate power supplies and electronics.
The high powers supplied by the fiber amplifiers raise two engineering difficulties. First, the fiber amplification process is inherently noisy, due to acoustic coupling of the fiber to the environment, electrical design of the amplifier, and spontaneous Brillouin scattering (SBS). 

The NUFERN MODEL NUMBER suffers from several problems: it is controlled via a USB interface, requiring a digital ground on the control PCB inside the device. Further, this digital ground is not isolated from the remainder of the electronics leading to potential ground loops between the controlling computer and the fiber amplifier. 
We solve this problem by replacing the USB control board with a custom-designed, ethernet based solution that is compatible with our custom control system. 
There are two power supplies in the fiber amplifier - a general purpose supply for the control electronics and first stage and a high current supply for the second stage. 
Further, despite the presence of a water-cooled coldplate to cool the fiber and pump diodes, the high power supply is air cooled using a fan approximately 20 cm away from the gain fiber and has been thought to lead to increased noise at acoustic frequencies.
To reduce acoustic coupling and reduce switching power supply noise, both power supplies were removed from the fiber amplifier enclosure, the low power supply was replaced by acopian XXXX PSU and the high power supply was filtered using an XXX line filter.
Combined, these changes serve to suppress certain noise spurs measured in the relative intensity noise (RIN) of the fiber amplifier, as seen in figure 1. 
The measurements were carried out on two fiber amplifiers from the same manufacturing batch, one with the modifications and one without.


\section{Mode and beam shaping}
The free-space propagation of the fiber amplifier output presents several challenges (1) the high power is sufficient to induce substantial thermal lensing, (2) the positional fluctuations of the waist at the atoms cannot exceed 
a few microns from shot to shot, (3) by construction, the beam is retroreflected, which threatens the lifetime of the amplifier unless sufficient isolation is used (4) approximately 1 W of power leaks into the undesired cladding modes of the fiber. 
To condition the beam into a desired shape, a train of optical elements is used, shown in figure \ref{fig:opticalelements}, which shapes the beam. 
Due to the high laser power, fused silica glass is used everywhere but in the acousto-optic modulator, Berek compensator and optical isolators, to minimze thermal lensing.
For the same reason, IBS coatings are used wherever possible, due to the higher damage thresholds.
The large-diameter fiber tip is mounted in a monolithic, oxygen-depleted copper mount, and the beam is collimated using an XXX 20 mm fused-sileca triplet collimator from Opto-Sigma, which produces a beam with $m^2 = XXX(X)$.
Then, the polarization is cleaned up using an IBS coated Brewster plate polarizer, which has the added benefit of rejecting the undesired cladding modes. 
The fact that the lattice is retro-reflected, combined with the observation that optical isolation falls with applied optical power\cite{LIGO} means that two stages of optical isolators must be used. The optical isolators are based on the 5 mm diameter Faraday rotators from Thorlabs. 
The optical isolators are located approximately 1 m away from the ultracold atomic system, meaning that undesired stray fields can have a dramatic effect.
To overcome this problem the isolators are enclosed in mu-metal shielding, reducing the field outside the can by an estimated factor of XXX.

\section{Feedback}

With an appropriately shaped and isolated beam, it is now important to consider two aspects of the desired experiment: (1) the lattice power must be continuously tunable from around 10 $\mu$W to around 20 W, (2) the location of the minimum gaussian beam waist must be accurately positioned to overlap the other lattices and dipole traps present in the experiment, and must remain there for the duration of the collection process (typically hours).
To appease the first of these requirements, a two-stage feedback system is implemented. The reason for the two stages is simple: the experiment typically operates in one of two regimes, which we will term \textit{detection} and \textit{interaction}. 
In the interaction phase of the experiment, the lattice is relatively shallow (depths of $\approx 10 E_R$, where $E_R$ is the geometric recoil of the lattice), allowing atoms to tunnel and interact with other atoms. 
In this regime, fine and potentially fast control is required.
In the detection phase, the depth of the lattice is dramatically raised ($\approx 2000 E_R$), isolating atoms in their individual wells so that Raman sideband imaging can be performed \cite{parsons2015}.
In this regime, the control need not be fast, and the passive stability of the Mephisto laser ensures that noise is low at relevant frequency scales.
Thus, two loops are utilized, a fast loop that uses an acousto-optic modulator to actuate the laser power at low powers, and a slow loop which uses a Berek compensator, to actuate in the high power mode.

After the isolation stage, we use the sequence of optical elements shown in figure \ref{fig1}, consisting of a polarizer, half-wave plate (HWP), Berek compensator mounted on a precision galvo (Thorlabs XXX) half-wave plate, and polarizer. The berek compensator is a simple $z$-cut quartz plate coated to be anti-reflecting at 1064 nm. 
By tilting the plate about its vertical axis, the extraordinary axis is mixed into the propagation of the beam, leading to a tilt-dependent bi-refringence that can be tuned from a zero-wave plate past a half-wave plate.
Combined with the surrounding polarization optics, rotation of the Berek compensator changes the transmitted power with perfect contrast and transmitted power as a function of the tilt angle is shown in figure \ref{fig2}.
If the half-wave plates in the Berek compensator are tuned to precisely 45 degrees from the angle of the tilt, the modulation contrast of the transmission is perfect, however, we purposefully detune from this configuration such that the minimum power transmitted through this system is approximately 1 W for each lattice axis. 
Thus, when the system is in \textit{interaction} mode, tilt noise on the Berek compensator is strongly suppressed by positioning the system at the minimum.

Past the Berek compensator we use a water-cooled, high power TeO$_2$ AOM (part number XXXX), where the power of the RF supplied to the AOM is used to actuate the fast, low power feedback loop.
The local oscillator (LO) RF source is a custom-built printed circuit board (PCB) using a phase locked loop (PLL) based on the ADF4002 IC and XXX voltage controlled oscillator (VCO), preamplified with the XXX low noise amplifier. 
This RF power is actuated using a mixer (Minicircuits XXX), which functions down to DC and a high-isolation RF switch (Minicircuits XXX), followed by a power amplifier stage (RF-bay XXX). After the AOM, the beam is sampled twice using a single beam sampler (Thorlabs BSF-10C), by taking the reflections from the front and rear facets of the optic, which are conveniently angled away from each other by XXX.
One of the sampled beams is further attenuated by a factor of approximately 4\% by reflecing off an uncoated fused-silica plate, and the two sampled beams are directed toward a pair of identical photo-diodes. 

The design of the photodiodes is exceedingly important for both the continued successful operation of the feedback loops as well as the characterization of the performance. The typical feedback loop used in the experiment is based on a simple trans-impedance amplifier design, combined with a low noise, linear, small active area photodiode. MORE DESIGN STUFF HERE



\section{Photodiodes}
FIGURE: Transimpedance amplifier, with our parameters
\begin{itemize}
	\item Transimpedance amplifier - typical parameters
	\item design criteria, part numbers, typical voltage levels
	\item cutting off windows
	\item thermal stuff (inGas)
	\item lens or no
	\item using photodiodes for looking at these thing
\end{itemize}

\subsection{Residual Intensity Noise}
RIN:
\begin{itemize}
\item what is it and how do you measure it
\item easy to measure, enters into parametric heating effects
\end{itemize}

\subsection{Heating Effects}
\begin{itemize}
	\item Trap shaking, parametric heating
\end{itemize}

\section{Lattice Beam Optics}


\begin{itemize}
\item evidence of fiber tip shaking if there were fans inside
changing collimator dist wrt fiber - focus about 80 cm - focus through aom - no movable optics between AOM and chuck
\item mode shape changes dramatically with pump current (output power)
\end{itemize}

\subsection{Isolation}
\begin{itemize}
\item backreflections are bad,
\item retroreflection -> 2 stages
\item as seen in ligo Yttg isolation decreases with optical intensity - coupling between optical effects and faraday
\item not as bad as ligo, since not in vacuum.
\item thorlabs -> high power 5mm aperture -> BK7 bad
\item waveplate ibs coating
\item Can of mumetal for physics reasons
\item attenuation of field is XXXXX
\item watercooling
\item stack 2x
\end{itemize}

DUMPING
-watercooled beamdumps far removed from the optics for thermal reasons
\subsection{Intensity Control}
2 Step intensity control

\subsubsection{Berek Compensator}
\begin{itemize}
\item high dyn range actuator
\item berek compensator
\end{itemize}

custom coating quartz plate that is z-cut mounted on camtech galvo
can act as translates rotation into polarization rotation
with brewster polarizers can act as variable attenuation
bias with waveplates
BW = XXX
attenuation = XXX
induces walkoff

BEAM with well defined polarization on optical table 
$M^2$ = this, waist focused 80cm to a 650um, which is consistent with ABCD tracing

\subsubsection{AcoustoOptic Modulation}

\begin{itemize}
    \item AOM is this 80MHZ, TeO2
    \item large active area AOM to reduce intenisty going through
    \item thermal problems
    \item watercooled
    \item custom flexure mount to minimize motional degrees of freedom.
    \item obtain a diffraction efficiency of $M^2$
    \item 20cm in front of AOM - diffraction process changes this - characterize this
    \item AOM's thermal effects pointing noise mostly. BW is traveling wave such and such speed of sound.
    \item shut off time is 500ns
    \item polarization is negligible
    \item POINTING is stable.
\end{itemize}

\section{Feedback}
\begin{itemize}
    \item aom
    \item AR pickoff
    \item PD - lenses and imaging
    \item Loop filter sqrt
    \item Exponential
    \item Pump current
\end{itemize}





\subsection{Acousto-optic Modulation}

FIGURE - RIN OF MEPHISTO, driven by difft oscillators
Mephisto+oscillators
\begin{itemize}
	\item how does the RIN spectrum change as a function of oscillators
\end{itemize}
\section{Fiber Amplifier system}
FIGURE - BLOCK DIAGRAM OF FIB AMP SYSTEM

\subsection{Fiber Amplifier Modifications}
FIGURE - SBS DATA
\begin{itemize}
	\item Mephisto+Fiber amplifiers
\item fiber amplifiers work like so so
\item schematic what it looks like and does
\item Modifications
	\begin{itemize}
		\item electronics control board,
		\item PSU's
		\item filters
		\item switching spikes removed
		\item fan noise stopped
	\end{itemize}
\item Two stage amplifier
	\begin{itemize}
		\item second stage has this special fiber 
		\item large mode area fibers
		\item leakage into cladding modes 
		\item sbs (cut off)
	\end{itemize}
\item SBS data
\end{itemize}
\section{Telescope}
-beam parameters such and such

\section{Alignment Tricks}


\end{document}
